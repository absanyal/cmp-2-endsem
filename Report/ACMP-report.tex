\documentclass[]{report}
\usepackage{graphicx}
\usepackage{amsmath}
\usepackage{amsfonts}
\usepackage{amssymb}


% Title Page
\title{Advanced Solid State\\
End-Semester Report}
\author{}


\begin{document}
\maketitle

\section{Q1}
Attached.

\section{Q2}
Attached.

\section{Q3}
I have solved the non-interacting band with both real and $k$-space Bloch Hamiltonians, with identical result.
\begin{figure}[h!]
	\centering
	\includegraphics[width=0.7\linewidth]{non-interacting}
	\caption{Non-interacting band structure.}
	\label{fig:non-interacting}
\end{figure}

The band structure shift due to perturbation theory is shown as well.

\begin{figure}[h!]
	\centering
	\includegraphics[width=0.7\linewidth]{pert}
	\caption{Correction due to perturbation.}
	\label{fig:pert}
\end{figure}

The HF code is attached. Due to some bug, which could not be traced, the energies are not converging.

It is to be noted that the value of the correction is highly dependent on the number of atoms in the unit cell as the number of lattice points chosen for doing the calculation.

\section{Q4}
Working code has been attached.

\section{Q5}
The screening constant, $\lambda$, has been estimated from
\begin{align}
\lambda = \sqrt{\frac{3n}{2 E_f}}
\end{align}
where $n=2$ is the number density of electrons and $E_f$ is the Fermi energy, estimated from the band structure.

\begin{figure}[h!]
	\centering
	\includegraphics[width=0.7\linewidth]{screening}
	\caption{Correction due to screening.}
	\label{fig:screening}
\end{figure}

\section{Q6}
In order to calculate the ferromagnetic constant, $J$, the Bloch Hamiltonian was first diagonalized, and then, the Wannier function was calculated using the following.

\begin{align}
W_{m,R_n}(x) = \sum_k u_{m}(x)\exp(i k x) \exp(-i k n a)
\end{align}
It was confirmed that the exchange integral remains constant.

\begin{figure}[h!]
	\centering
	\includegraphics[width=0.7\linewidth]{wannier}
	\caption{Wannier functions for ground and first excited state for the same certain unit cell.}
	\label{fig:wannier}
\end{figure}

\begin{figure}[h!]
	\centering
	\includegraphics[width=0.7\linewidth]{spin-wave-disp}
	\caption{Spin wave dispersion for first two bands.}
	\label{fig:spin-wave-disp}
\end{figure}


\end{document}          
